%Roman to Greek map
\newcommand{\Sa}{\alpha}
\newcommand{\Sb}{\beta}
\newcommand{\Sc}{\gamma}
\newcommand{\Sd}{\delta}
\newcommand{\Se}{\epsilon}
\newcommand{\Sf}{\xi}
\newcommand{\Sg}{\phi}
\newcommand{\Sh}{\eta}
\newcommand{\Si}{\iota}
\newcommand{\Sj}{\kappa} %?
\newcommand{\Sk}{\kappa}
\newcommand{\Sl}{\lambda}
\newcommand{\Sm}{\mu}
\newcommand{\Sn}{\nu}
\newcommand{\So}{\omicron}
\newcommand{\Sp}{\pi}
\newcommand{\Sq}{\theta}
\newcommand{\Sr}{\rho}
\newcommand{\Ss}{\sigma}
\newcommand{\St}{\tau}
\newcommand{\Su}{\upsilon}
\newcommand{\Sv}{\nu} %?
\newcommand{\Sw}{\omega}
\newcommand{\Sx}{\chi}
\newcommand{\Sy}{\psi}
\newcommand{\Sz}{\zeta}

%\newcommand{\MLmap}[1]{\IfStrEqCase{\MakeUppercase{#1}}{
%						{A}{\MLa}}[OHNO]}

\renewcommand{\S}[1]{\IfStrEqCase{#1}{
						{a}{\Sa}
						{b}{\Sb}
						{c}{\Sc}
						{d}{\Sd}
						{e}{\Se}
						{f}{\Sf}
						{g}{\Sg}
						{h}{\Sh}
						{i}{\Si}
						{j}{\Sj}
						{k}{\Sk}
						{l}{\Sl}
						{m}{\Sm}
						{n}{\Sn}
						{o}{\So}
						{p}{\Sp}
						{q}{\Sq}
						{r}{\Sr}
						{s}{\Ss}
						{t}{\St}
						{u}{\Su}
						{v}{\Sv}
						{w}{\Sw}
						{x}{\Sx}
						{y}{\Sy}
						{z}{\Sz}
						{A}{\Sa}
						{B}{\Sb}
						{C}{\Sc}
						{D}{\Sd}
						{E}{\Se}
						{F}{\Sf}
						{G}{\Sg}
						{H}{\Sh}
						{I}{\Si}
						{J}{\Sj}
						{K}{\Sk}
						{L}{\Sl}
						{M}{\Sm}
						{N}{\Sn}
						{O}{\So}
						{P}{\Sp}
						{Q}{\Sq}
						{R}{\Sr}
						{S}{\Ss}
						{T}{\St}
						{U}{\Su}
						{V}{\Sv}
						{W}{\Sw}
						{X}{\Sx}
						{Y}{\Sy}
						{Z}{\Sz}
						}[OHNO]}

\newcommand{\pmodNew}[1]{\mkern4mu\left({\rm mod}\mkern 6mu#1\right)}

%Needed functions by other macros
\newcommand{\card}[1]{\left|#1\right|}
\newcommand{\floor}[1]{\left\lfloor #1 \right\rfloor}
%Sequences
\newcommand{\Seq}[1]{{\bf{\MakeLowercase{#1}}}}
\newcommand{\SeqE}[2]{\Seq{#1}_{#2}}
%\newcommand{\SeqS}[1]{{\overline{\S{#1}}}}
\newcommand{\Tuple}[1]{\left( #1 \right)}
\newcommand{\Tup}[1]{{\bf{#1}}}
\newcommand{\TupE}[2]{\Tup{#1}_{#2}}
\newcommand{\setCard}[1]{\card{\SetL{#1}}}
\newcommand{\tupCard}[1]{\card{\Tup{#1}}}
\newcommand{\Sequence}[1]{\left\langle #1 \right\rangle}
\newcommand{\seqCard}[1]{\card{\Seq{#1}}} 
\newcommand{\union}{\cup}
\newcommand{\bigunion}{\bigcup}
\newcommand{\concat}{\sqcup} 						% Vector concatenation
\newcommand{\bigconcat}{\bigsqcup}
\newcommand{\cart}{\otimes}
\newcommand{\ssum}{\oplus}                          % Sequence sum
\newcommand{\bigssum}{\bigoplus}

%Vector-related info
\newcommand{\Vector}[1]{\left(#1\right)} 			% Explicit vector with elements #1

%Tensor-related info
\newcommand{\DataTensor}{\T{A}} 					% Tensor data to distribute
\newcommand{\tenOrder}{M} 							% Tensor order
\newcommand{\tenShape}{\Tup{I}}
\newcommand{\tenShapeProd}{\bar{I}}
\newcommand{\tenLocE}[1]{i_{#1}}
\newcommand{\tenLoc}{\Tup{i}}
\newcommand{\tenLinLoc}{\hat i}
\newcommand{\tenDim}[1]{I_{#1}} 					% Dimension of mode #1
\newcommand{\range}[1]{\mathcal{R}\left(#1\right)} 	% Range of mode #1
\newcommand{\rangeExpand}[1]{\Set{0,1,\dots,\tenDim{#1}-1}} 	% Range of mode #1

%Grid-related info
\newcommand{\Grid}{\T{G}} 							% Processing grid
\newcommand{\gridOrder}{N} 							% Processing grid order
\newcommand{\gridShape}{\Tup{P}} 					% Processing grid shape
\newcommand{\gridDim}[1]{P_{#1}} 					% Dimension of mode #1 of processing grid
\newcommand{\Proc}[1]{\ifstrequal{#1}{}{\Tup{p}}{\Tup{p}_{#1}}}
\newcommand{\GridLoc}[1]{\Grid_{\Tuple{#1}}}
\newcommand{\nComms}{{\hat c}}
\newcommand{\commInstance}{c}

%Distribution-related info
\newcommand{\Set}[1]{\left\{#1\right\}}
\newcommand{\SetE}[2]{{\rm \MakeLowercase{#1}}_{#2}}
\newcommand{\modeDist}[1]{\ifstrequal{#1}{}{\OrderedSetL{D}}{\OrderedSetL{D}^{\left(#1\right)}}} 			% Distribution of mode #1 of tensor
\newcommand{\modeDistFinal}[1]{\ifstrequal{#1}{}{\overline{\OrderedSetL{D}}}{\overline{\OrderedSetL{D}}^{\left(#1\right)}}}
\newcommand{\modeDistS}[1]{\mathcal{D}_{\modeDist{#1}}}
\newcommand{\modeDistSFinal}[1]{{\bar \modeDistS{#1}}}	
\newcommand{\modeDistexp}[1]{\Tuple{#1}}			% Explicit mode distribution with elements #1
\newcommand{\modeDistSexp}[1]{\mathcal{D}_{\Tuple{#1}}}	
\newcommand{\modeDistExample}{M} 					% Mode distribution for example purposes
\newcommand{\tenDist}[1]{\ifstrequal{#1}{}{\modeDist{}}{\modeDist{}^{(#1)}}} 				% Tensor Distribution
\newcommand{\tenDistFinal}[1]{\ifstrequal{#1}{}{\overline{\modeDist{}}}{\overline{\modeDist{}}^{(#1)}}}
\newcommand{\tenDistexp}[1]{\left[#1\right]} 	    % Explicit tensor distribution with elements #1

\newcommand{\fullElemSet}{\range{\tenShape}}
\newcommand{\fullProcSet}{\range{\gridShape}}
\newcommand{\elemSetNew}[2]{\SetL{I}^{\left(#2\right)}\left(#1\right)}
\newcommand{\elemSetI}[1]{\SetL{I}^{\left(#1\right)}}
\newcommand{\elemSetF}[1]{\SetL{F}^{\left(#1\right)}}
\newcommand{\elemSetR}[1]{\SetL{R}^{\left(#1\right)}}
\newcommand{\elemSet}[1]{\SetL{L}^{\left(#1\right)}}
\newcommand{\elemSetN}[2]{\SetL{F}\left(#1\right)^{#2}}
\newcommand{\ProcSet}{\SetL{P}}
\newcommand{\procVar}{k}
\newcommand{\procVarTwo}{h}
\newcommand{\subrangeNew}[3]{\SetL{I}^{\left(#2\right)}_{#3}\left(#1\right)}		% Used to define subranges
\newcommand{\subrangeR}[1]{\elemSetR{#1}}		% Used to define subranges
\newcommand{\subrangeI}[2]{\elemSetI{#1}_{#2}}		% Used to define subranges
\newcommand{\subrangeF}[2]{\elemSetF{#1}_{#2}}		% Used to define subranges
\newcommand{\subrange}[2]{\elemSet{#1}_{#2}}		% Used to define subranges
\newcommand{\subrangeN}[3]{\SetL{L}\left(#1\right)^{#2}_{#3}}
\newcommand{\subrangeFilter}[3]{\Set{ i \in \range{\tenDim{#1}} \middle|\quad i \equiv {#2} {\pmodNew {#3}}}}
\newcommand{\subrangeFilterExp}[3]{\Set{ i \in #1 \middle|\quad i \equiv {#2} {\pmodNew {#3}}}}
%Required functions
\newcommand{\vecfilter}[2]{#1\left(#2\right)} 				% Filter #2 applied to vector #1
%\newcommand{\vecfilter}[2]{{\rm filter}\left(#1,#2\right)} 				% Filter #2 applied to vector #1
\newcommand{\numElem}[1]{numElem(#1)} 				% Number of elements in Shape #1
\newcommand{\multiTolinearName}{{\rm multi2linear}\xspace{}}
\newcommand{\multiTolinear}[2]{\multiTolinearName\left(#1, #2\right)}
\newcommand{\multiTolinearShortName}{{\rm multi2linear}\xspace{}}
\newcommand{\multiTolinearShort}[3]{\multiTolinearShortName\left(#1, #2, #3\right)}
\newcommand{\linearTomultiName}{{\rm linear2multi}\xspace{}}
\newcommand{\linearTomultiShortName}{{\rm linear2multi}\xspace{}}
\newcommand{\linearTomulti}[2]{\linearTomultiName\left(#1, #2\right)}
\newcommand{\linearTomultiShort}[3]{\linearTomultiShortName\left(#1, #2, #3\right)}

\newcommand{\serProdName}{{\rm prod}\xspace{}}
\newcommand{\serProd}[1]{\serProdName\left(#1\right)}
\newcommand{\serProdShortName}{{\rm prod}\xspace{}}
\newcommand{\serProdShort}[2]{\serProdShortName\left(#1, #2\right)}

\newcommand{\tenIndVar}{k}
\newcommand{\procIndVar}{p}
%Collective proof info
\newcommand{\data}[2][]{{}^{#1}\mathcal{d}^{(#2)}}

\newcommand{\commModeDist}{\OrderedSetL{C}}
\newcommand{\commModeDistTwo}{\OrderedSetL{B}}
\newcommand{\commMode}{m}

\newcommand{\permuteVec}{\V{\pi}}
\newcommand{\permute}[2]{#1\left(#2\right)}
\newcommand{\invPerm}[2]{#1^{-1}\left(#2\right)}

\newcommand{\tempRank}{{\hat g}}

\newcommand{\nProcsComm}{\hat p}
\newcommand{\procRank}{r}
\newcommand{\rootRank}{o}

%Tammy commands
\newcommand{\Range}[1]{\mathcal{R}(#1)}
\newcommand{\RangeExpand}[1]{\set{0,1,\dots,#1-1}}
\newcommand{\ModeMaps}{\mathcal{D}}
\newcommand{\ModeMap}[1][m]{\mathcal{D}^{(#1)}}
\newcommand{\ModeMapElem}[2][m]{d^{(#1)}_{#2}}
\newcommand{\ModeSet}{\mathcal{\hat D}}
\newcommand{\ModeSetComp}{\ModeSet^{\circ}}